\chapter{Related Work}
\label{sec:rel}

\section{Predictive Compliance Monitoring in Process-Aware Information Systems: State of the Art, Functionalities, Research Directions}
\textbf{Rinderle-Ma et al.} \cite{rinderle-maPredictiveComplianceMonitoring2023}
\textbf{Keywords}: Predictive Compliance Monitoring, Predictive Compliance Monitoring System, Predictive Process Monitoring, Systematic Literature Review, Research Directions
\textbf{//Motivation:}
\begin{itemize}
\item "Business process compliance is\textbf{ a key area of business process management}"
\item and aims at ensuring that processes obey to compliance constraints such as regulatory constraints or business rules imposed on them.
\item "Process compliance can be checked during"
\begin{itemize}
\item "process design time based on verification of process models"
\item "at runtime based on monitoring the compliance states of running process instances"
\end{itemize}
\item "Compliance Monitoring (CM)(9, 10) is an integral part for monitoring and managing business processes in changing, complex regulatory environments such as the financial domain." 
\item "Yet, reactive management through auditing is still most prominent in compliance management of companies,"
\item "Building a PCM system is a complex task, resulting from a multitude of compliance constraints stemming from different and constantly changing regulatory documents [12]"
\item In particular, research should avoid assuming the constraints to be readily available and stated in some logic, but often have to be extracted and updated based on regulatory documents [158].
\item "Life cycle handling" of process with adaption to (legal) constraints
\item "Typically, constraints are stated in natural language and scattered across multiple regulatory documents (cf. Ex. 1)."
\item "Moreover, this work assumes that compliance constraints are formalized u\textbf{sing some notion}." --> Darüber hinaus wird in dieser Arbeit davon ausgegangen, dass Konformitätsbeschränkungen mit Hilfe eines Begriffs formalisiert werden
\end{itemize}

\textbf{//Contribution}
\begin{itemize}
\item "This work, hence, analyzes existing literature from compliance monitoring as well as predictive process monitoring and provides an updated framework of compliance monitoring functionalities. Moreover, it raises the vision of a comprehensive predictive compliance monitoring system that integrates existing predicate prediction approaches with the idea of employing PPM with different prediction goals such as next activity or remaining time for prediction and subsequent mapping of the prediction results onto the given set of compliance constraints (PCM)."
\item "By combining the respective capabilities of PPM and CM, research can offer companies a means to proactively assess and manage their business processes with respect to future outcomes, compliance status, and risks."
\item "Predicate prediction – which is mainly followed by existing approaches, e.g., [16] – encodes each compliance constraint as prediction goal into prediction models."
\item "Yet, especially in combination with updating compliance violations,\textbf{ an open challenge remains how to define and update }the compliance degree while new events arrive throughout the event stream and to predict compliance states of single instances."
\end{itemize}

\textbf{//Discussion:}
\begin{itemize}
\item "Update 3: Continuous update of prediction results and compliance violations" --> ""Life cycle handling" of process with adaption to (legal) constraints" --> PCM (Predictive Compliance Monitoring)
\end{itemize}

\section{Mining Process Models from Natural Language Text: A State-of-the-Art Analysis}
\textbf{Riefer et al.} \cite{rieferMiningProcessModels2016a}

\textbf{//Motivation}
\begin{itemize}
\item "Workflow projects are time-consuming processes."
\item "knowledge extraction and the creation of process models."
\item "necessary information is often available as textual resources."
\item "process model mining from natural language text has been a \textbf{research area of growing interest.}"
\item "Organizations are constantly trying to \textbf{analyze and improve their business processes.}"
\item "This is only possible if the knowledge about the processes is available \textbf{in a structured form like a business process mode}l."
\item "85\% of the knowledge and information are estimated to be available in an unstructured form, mostly as text documents (Blumberg and Atre 2003) \cite{blumberg2003problem}".
\item "It gives {people with no knowledge about process modeling the possibility to create process models,} which is an important goal in view of the fact that structured data becomes more important."
\item "Text Mining approaches have been developed for UML class diagrams (Bajwa and Choudhary 2011), entity relationship models (ERM) (Omar et al. 2008) or business process models (Friedrich et al. 2011). Current approaches do n\textbf{ot aim at replacing an analyst but at helping him to create better models in less time}."
\item \textbf{Goal of this paper:} "general overview in terms of a state-of-the-art analysis is missing"
\item 
\end{itemize}

\textbf{//Introduction -> Structure of Thesis:}
\begin{itemize}
\item "The used research methodology and the identification of the relevant literature are presented in section 2, while section 3 introduces the most important methods and terms for processing natural language texts. Section 4 gives an overview of current approaches. The comparative analysis, which consists of a detailed theoretical analysis and a proposed practical analysis, is conducted in section 5. The results are discussed in Section 6, followed by a conclusion in section 7."
\item 
\item 
\end{itemize}

\textbf{//Methodology}
"To define the current state of research and to identify different approaches for mining process models from natural language text, a systematical literature review was conducted. The three literature databases Google Scholar, SpringerLink and Scopus were used for the research. The following search keywords were derived from the title and thematic of this paper: natural language processing, process model, process modeling or process model generation, process model discovery, text mining, process mining and workflow. These were used in various combinations. As the used keywords cover broad research areas, they lead to a high number of search results. That complicated the identification of the relevant literature. The search results were checked through a title and abstract screening to identify the relevant work. Hence, only publications which explicitly mention text-to-model transformations were considered as relevant. There turned out to be a problem with the author’s way of describing their work: instead of referring to text mining or natural language processing, they often used the text type, such as use cases or group stories, to outline their work. The search in a database was aborted when a significant amount of repetitions or loss of precision was noticed. It turned out, that the keyword search provided a low degree of relevant papers. Hence the keyword search was skipped and changed to a cross reference search. Table 1 shows the literature search results."

\textbf{Table 1: Results of literature research}


Afterwards, further works were detected through a backwards search. The work of \textbf{(Leopold 2013)} provided a proficient starting point. The focus was set on approaches which generate a business process model. Five appropriate approaches could be identified:

BPMN model from text artefacts (Ghose et al. 2007) 
BPMN model from group stories (Goncalves et al. 2009) 
BPMN model from use cases (Sinha and Paradkar 2010
BPMN model from text (Friedrich et al. 2011) 
Model from text methodologies (Viorica Epure et al. 2015)

\textbf{Design‑time business process compliance assessment based on multi‑granularity semantic information }
Xiaoxiao Sun
\cite{sunDesigntimeBusinessProcess2023}

\begin{itemize}
\item Business Process Compliance (BPC) is an essential part of BPM that measures how effectively an organization’s business processes comply with all relevant laws, regulations, guidelines, and standards [2].
\item Examples of critical regulatory documents are the Health Insurance Portability and Accountability Act (HIPAA), the Sarbanes-Oxley Act (SOX), and the General Data Protection Regulation (GDPR) [3]. 
\item Companies that violate these regulatory documents risk losing the trust of investors, incurring financial fines, and facing criminal charges. As a result, adhering to rules from multiple sources has become essential for every organization to avoid huge fine losses and also to improve business process transparency [4].
\item However, in company’s practice, checking and ensuring the consistency of the organization’s business processes with regulatory documents, i.e., BPC checking, is still largely done manually with lots of efforts.
\item In addition, the costs of manual review increase significantly due to the constant changes in regulatory environment [5, 6].
\item \textbf{However, formal languages often posed challenges in terms of comprehension [9].}
\item In this study, we focus on parsing BPMN, which is one of the most widely used notations for business processes with the latest version of 2.0 [28].
\item BPMN model comprises three types of nodes: events, activities, and gateways. Events, represented by circles, indicate occurrences within the process. Activities, depicted as rounded rectangles, represent tasks performed within the business process. Gateways, shown as diamonds, control the flow of the business process. However, the proposed approach is generic and can be applied to other modeling languages.
\end{itemize}





