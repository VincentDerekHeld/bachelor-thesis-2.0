\chapter{Related Work}
\label{sec:rel}

	In order to learn the current state-of-the-art methods of auto-generating business process models and thus answer the research question comprehensively, a systematic literature review must be performed so that what kind of efforts are made can be learned as well as what are the most preferred techniques and what open challenges exist. The literature review is conducted under the guidance of Kitchenham et al. given in their paper \cite{literature_review_guidance}. The work consists of several stages: Firstly, the electronic database used to run the search is chosen. Then the selection criteria are defined, and articles are filtered accordingly. After that, a horizontal search will be run to cover as many papers as possible. Finally, a list of the final literature is studied carefully, and helpful information is extracted.
	
% search string
	To perform a comprehensive literature review, three most famous electronic databases are chosen, i.e., IEEE, Springer, and ACM. Nevertheless, only using these three databases, There is still a minor chance that some important articles will be missed. Therefore, Google scholar was also used as a complement because it covers a wide range of literature, from conference papers to degree theses. The search string used for the literature review is developed using two phrases, which are the most important ones for our research: \textit{business process model} and \textit{natural language processing}.
	
% selection criteria
	In the next step, inclusion and exclusion criteria should be defined. They describe a list of desired and undesired features for the literature selection to obtain relevant studies and support our research and future work. Inclusion criteria were developed as follows: \textbf{IC-1}: NLP should have high relevance to the research paper. \textbf{IC-2}: BPMN should have high relevance to the research paper. \textbf{IC-3}: The research paper should describe the generation of the BPMN model using NLP. Exclusion criteria were: \textbf {EC-1}: the research paper is not written in English. \textbf{EC-2}: The research paper is not in the form of a proper scientific article. 
	
		\begin{table}[]
		\begin{center}
		\caption{\centering Overview of Systematic Literature Review Protocol}
		\begin{tabular}{lccl}
    	\textbf{Database}\hspace{30mm} & \textbf{hits} & \textbf{selected} &  \\
    	\hline
		IEEE                     		& 56   & 5   &      		\\
		Springer                 		& 275  & 8   &      		\\
		ACM                      		& 201  & 2   &      		\\
		Google scholar           		& 31   & 3   &      		\\
		\hline
		Result horizontal search	 	& 563  & 18  &      		\\
		Vertical search          		&      & 4   &  \hspace{5mm}add papers  \\
		\hline
		Overall                  		&      & 22   &     
		\end{tabular}
		\end{center}
		\floatfoot{number of hits using the search string in different databases.}
	\end{table}
	
	During the literature selection, the first step was to identify duplicates since multiple electronic databases were used. Duplicates refer to articles that have the same title and authors. In the next step, the article's title, abstract and introduction parts were read and the inclusion and exclusion criteria were applied to shape the final result further. Finally, the whole article was read and then a vertical search was performed to identify the related papers used in our selected papers. As a final result, 17 papers were chosen. 
	
	Among the chosen papers, \cite{literature_review_1} \cite{literature_review_2} \cite{literature_review_3} \cite{literature_review_4} are literature reviews that analyzed the development and usage of process model generation methods. \cite{literature_review_3} points out that the NLP is the most widely adopted method and it can also be combined with other methods to increase accuracy. \cite{literature_review_1} and \cite{literature_review_3} give a list of tools for NLP and process model generation that have been used in previous works. \cite{literature_review_4} compares several papers using NLP to extract process models with different inputs and concludes the typical steps that have to be performed. \cite{complement_2} and \cite{complement_3} propose their findings in identifying the inconsistencies between the textual description and the generated process model. \cite{complement_4} however proposed a new part-of-speech tagging method that is specifically trained for business process management, which can effectively reduce the error rate in grammatical tagging.
	
	A novel breakthrough is made in the work of \cite{t2m_1}, where a method was developed which extracts information from textual descriptions to automatically generate the business process models regardless of the structure of the input text. The paper gives an excellent overview of the steps that should be executed during the process model extraction and the potential challenge one might encounter. The Authors performed three vital steps to process the text input: (i) syntax parsing using the part-of-speech tagging method, (ii) semantic analysis using FrameNet and WordNet, and (iii) anaphora resolution. Finally, they can generate a process model based on the data. Nevertheless, their work was published in 2012, and some of the techniques could already be outdated. We see great potential here to work on an updated approach to improve the results of this work based on the paper.
	
	 Some limitations of \cite{t2m_1} are also addressed in \cite{pre_processing_1}: The textual description must be grammatically correct, otherwise the model will produce an incorrect output. Furthermore, the process in the description must develop sequentially and cannot contain examples or questions. Another work offered in \cite{t2m_2} focuses on the extraction of declarative process models to address the problem that many NLP models can only handle the imperative process models. This is done by introducing many grammatical constraints to analyze the relationship of words. This idea illustrates us to also expand the analysis of the semantic analysis so that our model is also able to deal with the textual documents that have a complex description of the process. 
	 
%	 Among all works, little effort is made to visualize the process model, \cite{complement_1} introduces a web-based NLP model extraction service. However, this work leverages the extraction model of others, and thus the output accuracy cannot be well guaranteed. As mentioned, our model should assist people who do not have a technical background in the process model generation. \cite{complement_1} gives us an idea of additionally implementing a web-based frontend that can ease the use of our tool.