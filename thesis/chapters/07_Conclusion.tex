\chapter{Conclusion}
\label{sec:conclusion}

In this thesis, we presented an approach to automatically extract information from textual descriptions and based on which construct Business Process Model and Notation (BPMN) diagrams in the format of images. To achieve information retrieval, several open-sourced NLP libraries are used. The syntax parser is used to parse the sentences with the corresponding Part-of-Speech tags and semantic dependencies, which will then be divided into several clauses for further processing. A series of algorithms will then be applied to the clauses to identify and extract the related process information, which will be stored in an intermediate data structure. Several algorithms will then examine the data structures to construct the proper BPMN elements and place them in the correct order. Finally, the generated result will be rendered into BPMN diagrams as images.

The evaluation shows that the proposed system achieves comprehensive information retrieval with little uncaptured activities. The result of the proposed system turns out to be promising. On the one hand, the proposed approach replicates the result of the state-of-the-art approach; on the other hand, this work also makes modifications and improvements to the state-of-the-art. Furthermore, this work also leverages the Large Language Model (LLM) \textit{GPT-4} to perform the adjustment and correction on the generated output. The evaluation result shows positive that the LLM is capable of identifying and removing the process irrelevant information in the generated output as well as correcting the wrongly mapped elements.

However, the problem of process model auto-generation is still considered unsolved. Many process irrelevant information, like examples, is also included in the output to achieve a high information retrieval rate. The rule-based system is unable to generalize or abstract the extracted information, which human modelers prefer. Moreover, the system may suffer from missing of specific rules, so the system's performance is limited. Therefore, the proposed approach is considered to be a complement tool for human modelers to ease and accelerate the BPMN modeling.

In the future, a precise syntax parser can increase the accuracy of the system. A user can customize and extend the given rules to achieve a better result for further improvement. The use of semantic analyzing tools could also benefit the system's precision by filtering the process irrelevant information. Using a specifically fine-tuned LLM for BPMN to assist with the generation of process models is also a possible scenario, where the LLM can perform the text preprocessing or adjust the generated output, which might yield a better result as sorely using LLM or rule-based system.

