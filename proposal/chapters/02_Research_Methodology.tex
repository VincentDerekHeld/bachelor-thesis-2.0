\chapter{Research Methodology}

Design science is a paradigm of real-world problem-solving by creating innovative artifacts. Therefore, Design science research tightly connected the IT artifact with the application domain. Furthermore, the need and desire to improve the current environment and methods motivate Design science research and therefore requires innovative artifacts to address such problems \cite{DSM_1}. We adopted the research methodology of \cite{DSM_2} here and followed the research process model given in their work.

\section{Problem centered approach}
Although some work in the current field was done, we wanted to develop better tools to automatically extract the business process model for the broad audience of end users, i.e., users within a business organization with little knowledge about business process modeling or underlying technologies. Such motivation provides us with an opportunity to work on creating the tool mentioned above. This problem-centered approach leads us to the first step of the research process, according to \cite{DSM_2}.

\section{Problem Identification And Motivation}
Modeling business processes requires experts with relevant knowledge and can be exhausting and time-consuming. Thus, small companies usually cannot afford such experts and business managers usually prefer to describe processes using natural language. As a result, an organization usually process a large amount of text documents  \cite{literature_review_2}. However, the business process provides an intuitional overview of the business process and can potentially increase a company's productivity.

\section{Objectives of a solution}
Our objective is to create an easy-to-use tool that uses the Nature language processing technology to automatically extract information from organizations' regulatory documents and generate business process models.

\section{Design \& development}
The development of the new artifact adopted the critical success chain (CSC) method, which uses literature to support and consolidate the conceptual basis of the artifact designing \cite{DSM_2}. We addressed the issues and the needs identified earlier, such as how to find a proper tool to extract information from regulatory documents or how to process such information to generate a BPMN model. We conducted a literature review and used the helpful information from the selected papers to combine their ideas and develop our own artifact. The intended artifact is to develop a prototype that leverages the NLP technique to automatically extract business process models from regulatory documents.

\section{Demonstration}
In the demonstration activity, we want to illustrate how we can use our new artifact to solve instances of problem \cite{DSM_3}. We plan to implement a web-based front-end so that customers can use our artifact rather easily, even if they have no knowledge of programming. The customer can enter the textual description which is a regulatory document into an input field on the website and click a button to send the request to convert the text to the BPMN model. Then the regulatory document will be sent to the backend and processed there. Finally, the customer will be able to see a BPMN model presented in an image on the website.

\section{Evaluation}
The evaluation phase is vital in the design of an artifact. The evaluation examines how well the designed artifact solves the problem, which involves comparing the the actual output of the problem and the output generated by the artifact \cite{DSM_3}. We intended to first manually generate business process models using several different kinds of documents. Then we will use our artifact with the same documents  




The evaluation measures how well the artifact supports a solution to the problem. This activity involves comparing the objectives of a solution to actual observed results from use of the artifact in context. Depending on the nature of the problem venue and the artifact, evaluation could take many forms. At the end of this activity the researchers can decide whether to iterate back to step three to try to improve the effectiveness of the artifact or to continue on to communication and leave further improvement to subsequent projects.



\section{Communication}
To ensure the delivery of the desired artifact, every aspect of the problem and the design of the artifact will be communicated and discussed with the relevant stakeholders \cite{DSM_3}. Since this is a bachelor's thesis, the primary contact is the author's advisor. Furthermore, we will also seek advice and suggestions from Prof. Dr. Stefanie Rinderle-Ma and the corresponding Chair of Information Systems and Business Process Management (i17).






