\documentclass[runningheads]{llncs}

\usepackage[T1]{fontenc}

\usepackage{graphicx}

\usepackage[colorlinks,bookmarksopen,bookmarksnumbered,citecolor=blue, linkcolor=blue, urlcolor=blue]{hyperref}

\usepackage[utf8]{inputenc}

\begin{document}
	\title{Proposal to Bachelor's thesis}
	\subtitle{Auto-generation of business process models using natural language processing}
	\author{Shuaiwei YU \inst{1} }
	\institute{Technical University of Munich\\
	\email{shuaiwei.yu@tum.de}}
	\maketitle

%todo: is there a need for abstract in the proposal?		
%	\begin{abstract}
%	\end{abstract}

	\section{Introduction}
%	bpmn introduce
	Business processes are fundamental elements for companies and organizations. They aggregate all the tasks, activities, and timelines involved in companies' workflow whose aim is to provide business or to create value \cite{literature_review_2}. Business Process Modeling Notation, also known as BPMN is a modeling language describing such workflows by using graphical notations and thus provides an easily understandable overview of the operations performed in the organization for all business users \cite{literature_review_1}. 
	
	Due to the importance of Business processes, leveraging the BPMN techniques can positively affect an organization's performance and thus increase its productivity. However, not everyone is familiar with the BPMN designing techniques. Consequently, managers, along with other process participants prefer using natural language to define the processes. As a result, organizations usually have a large amount of information stored as text documents \cite{literature_review_2}. There is a need to translate text documents into the process model regarding such a situation. However, process modeling is not a simple task, but is time-consuming and experts with professional knowledge are required. 
%todo: if there is a need to expand the details of difficulties of process modeling, then refer to literature_review_2 
	
%	nlp introduce
	Over the past years, the development of AI techniques brought solutions to many technical difficulties. Natural Language Processing (NLP), as one of the AI's branches, could possibly address the problem of the difficulties in process modeling. Natural Language Processing is an interdisciplinary discipline focusing on the study of algorithms that enable the computer to understand and process the human language\cite{main_1}. During the understanding and processing of the natural language text, NLP performs three types of analysis: Firstly, morphological analysis is performed, which analyze the structure of words. The syntactic analysis then explores the grammar relationship between words in sentences, deciding which grammar category the word belongs to. Finally, semantic analysis is executed, which leverages the afore analyses to define the meaning of the text based on the knowledge of sentence structure and the relationship between words \cite{literature_review_2}. 
%todo: if there is a need to expand the details of NLP analysis, then refer to literature_review_2 introduction part
	
%	why should nlp be used to generate the bpmn
	The unique features of the NLP technique make it very suitable for exploiting information from the text documents that record the firm's business process and then analyzing the data to generate the process models automatically. This paper serves as a proposal to suggest using NLP to extract the information from text written in nature language and automatically generate the corresponding business model.
	
%todo: should I write a reminder here? since the article is not long
		
	\section{Systematic Literature Review Protocol}
	In order to formulate a good research question and answer the research question comprehensively, a systematic literature review must be performed so that we can learn what kind of efforts are made as well as what are the most preferred techniques. The literature review is conducted under the guidance of Kitchenham et al. given in their paper \cite{literature_review_guidance}. The work consists of several stages: Firstly, the research question is formulated, and the electronic database used to run the search is chosen. Then the selection criteria are defined, and articles are filtered accordingly. After that, a horizontal search will be run to cover as many papers as possible. Finally, a list of the final literature is studied carefully, and helpful information is extracted.
	
	The main research question (\textbf{RQ}) is formulated as: "\textit{How can business process models be generated automatically using the Nature language processing technique?}". The main research question embeds several aspects: 
	
%todo: complete research question

 		 \textbf{RQ?}: "\textit{How can the generated business process model be visualized?}"
	
	\begin{table}[]
	\centering
	\begin{tabular}{lll}
    \textbf{Database}\hspace{50mm} & \textbf{hits} \hspace{10mm} & \textbf{selected} \\
    \hline
	IEEE                     & 56   & 0        \\
	Springer                 & 275  & 0        \\
	ACM                      & 201  & 0        \\
	Google scholar           & 0    & 0        \\
	\hline
	Result horizontal search	 & 10   & 5        \\
	Vertical search          & 0    & 0        \\
	\hline
	Overall                  & 0    & 0       
	\end{tabular}
	\end{table}
	
		
	\section{Systematic Literature Review Results}		
		
	\section{Design Science Research Methods}
	
	\section{Conclusion}
		
				
	\newpage
	\begin{thebibliography}{99}
	
	\bibitem{literature_review_1}
	Maqbool, B., Azam, F., Anwar, M. W., Butt, W. H., Zeb, J., Zafar, I., ... \& Umair, Z. (2019). A comprehensive investigation of BPMN models generation from textual requirements—techniques, tools and trends. In Information Science and Applications 2018: ICISA 2018 (pp. 543-557). Springer Singapore.
	
	\bibitem{literature_review_2}
	de Almeida Bordignon, A. C., Thom, L. H., Silva, T. S., Dani, V. S., Fantinato, M., \& Ferreira, R. C. B. (2018, June). Natural language processing in business process identification and modeling: a systematic literature review. In Proceedings of the XIV Brazilian Symposium on Information Systems (pp. 1-8).
	
	\bibitem{literature_review_3}
	Indahyanti, U., Djunaidy, A., \& Siahaan, D. (2022, October). Auto-Generating Business Process Model From Heterogeneous Documents: A Comprehensive Literature Survey. In 2022 9th International Conference on Electrical Engineering, Computer Science and Informatics (EECSI) (pp. 239-243). IEEE.
	
	\bibitem{literature_review_guidance}
	Kitchenham, B. (2004). Procedures for performing systematic reviews. Keele, UK, Keele University, 33(2004), 1-26.
	
	\bibitem{main_1}
	Sintoris, K., \& Vergidis, K. (2017, July). Extracting business process models using natural language processing (NLP) techniques. In 2017 IEEE 19th conference on business informatics (CBI) (Vol. 1, pp. 135-139). IEEE.
	
	\end{thebibliography}

	
\end{document}
